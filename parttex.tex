\documentclass[journal,sort]{IEEEtran}

\usepackage{lineno,hyperref}
\usepackage{bm}
\usepackage{cancel}
\usepackage{booktabs}
\usepackage{multirow}
\usepackage{array,chngpage}
\usepackage{subfigure}
\usepackage{amssymb}
\usepackage{graphicx}
\usepackage{amsmath}
\usepackage{bm}
\usepackage{diagbox}
\usepackage{cite}
\usepackage{threeparttable}
\usepackage{url}
\usepackage{setspace}
\usepackage{caption}
\DeclareMathOperator*{\argmax}{argmax}
\usepackage[linewidth=1pt]{mdframed}
\usepackage{lipsum}
\usepackage{booktabs}
\usepackage{color}
\usepackage{longtable}


\bibliographystyle{IEEEtran}

\begin{document}

\title{An adaptive video steganography using the intra CTU quadtree structure for HEVC}
\author{XXXX}
	

\maketitle

\begin{abstract}
In most High Efficiency Video Coding (HEVC) steganographic schemes, the cover object, such as motion vector and intra prediction mode, is coincident with H.264/MPEG-4 AVC. Consequently, the impact of new features introduced by HEVC on steganography security and capacity is still a question. In this paper, the steganographic properties of HEVC intra CTU quadtree structure are studied.A four layer quadtree coding alogorithm is proposed to embed the secret payload. While maintaining the coding efficiency as much as possible, we have designed an adaptive matching algorithm to minimize the  statistical detectability.

	
	
\end{abstract}	
	
	
\end{document}