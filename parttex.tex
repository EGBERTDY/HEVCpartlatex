\documentclass[journal,sort]{IEEEtran}

\usepackage{lineno,hyperref}
\usepackage{bm}
\usepackage{cancel}
\usepackage{booktabs}
\usepackage{multirow}
\usepackage{array,chngpage}
\usepackage{subfigure}
\usepackage{amssymb}
\usepackage{graphicx}
\usepackage{amsmath}
\usepackage{bm}
\usepackage{diagbox}
\usepackage{cite}
\usepackage{threeparttable}
\usepackage{url}
\usepackage{setspace}
\usepackage{caption}
\DeclareMathOperator*{\argmax}{argmax}
\usepackage[linewidth=1pt]{mdframed}
\usepackage{lipsum}
\usepackage{booktabs}
\usepackage{color}
\usepackage{longtable}


\bibliographystyle{IEEEtran}

\begin{document}

\title{Adaptive and Generative intra-frame steganography in HEVC video using the intra quadtree partition structure}
\author{XXXX}
	

\maketitle

\begin{abstract}
Intra-frame steganography in HEVC video is a challenge task in the field of video steganography. This paper proposed an adaptive intra-frame steganography method in HEVC video. In this paper, compared to the traditional intra-frame cover, the intra prediction modes, the intra-frame partition structure has proven to be a more efficient and secured intra-frame hidden carrier for HEVC video. A novel video steganography method based on intra quadtree coding structure is proposed to embed the secret payload. In the proposed method, four algorithms using different kinds of quad-tree partition structure are designed to make full use of unique HEVC intra-coding structure for high-efficient video steganography.To minimize the potiential statistical detectability, an adaptive matching scheme is designed to use appropriate steganographic algorithms for different video content.The proposed method is examined on HD video database with different resolution and video contents,and results are further compared with previous Intra-frame steganography method to confirm the effectiveness and advantages of this method.
The contribution of  this paper including:(1)

	
	
\end{abstract}	
\begin{IEEEkeywords}
		video steganography, HEVC, quad-tree partition structure, adaptive sheme
\end{IEEEkeywords}
	
\section{Introduction\label{intro}}

As an important method to ensure communication security in the network environment, Steganography has always been a key, cutting-edge research area. Steganography utilizes the characteristics of massive data exchange in the Internet, constructs a hidden channel that humans cannot detect, and transmits a large amount of secret information. This technology is both a new opportunity and a new challenge for the security communication.


One of the most challenge research task in steganography is video steganography. As a carrier of large capacity, high concealment, redundant space diversity and robustness, video has more theoretical research significance and value than image and audio steganography. Taking HEVC coded video as an example, the coding standard designed for high-definition video, the amount of data of high-definition video itself is huge, and its coding technology also brings a new coding domain steganography space, in technical complexity, algorithm security and hidden The diversity of write space is more suitable for security information steganography.


\section{Steganographic characteristic of HEVC intra quadtree partition structure}

\subsection{Intra partition scheme in HEVC}
\subsection{Distortion Analysis on visual quality}

\subsection{Distortion Analysis on coding efficiency}
\subsection{Distortion Analysis on Security}
\section{The proposed Steganographic method for HEVC}

\section{The generative algorithm}

\subsection{The adaptive matching scheme}
\section{Framework}



\section{Experimental results}
\section{conclusion}




	
	
	
	
	
	
	
\end{document}