\documentclass[journal,sort]{IEEEtran}

\usepackage{lineno,hyperref}
\usepackage{bm}
\usepackage{cancel}
\usepackage{booktabs}
\usepackage{multirow}
\usepackage{array,chngpage}
\usepackage{subfigure}
\usepackage{amssymb}
\usepackage{graphicx}
\usepackage{amsmath}
\usepackage{bm}
\usepackage{diagbox}
\usepackage{cite}
\usepackage{threeparttable}
\usepackage{url}
\usepackage{setspace}
\usepackage{caption}
\DeclareMathOperator*{\argmax}{argmax}
\usepackage[linewidth=1pt]{mdframed}
\usepackage{lipsum}
\usepackage{booktabs}
\usepackage{color}
\usepackage{longtable}


\bibliographystyle{IEEEtran}

\begin{document}

\title{An adaptive video steganography using the intra CTU quadtree partition structure for HEVC}
\author{XXXX}
	

\maketitle

\begin{abstract}
In most High Efficiency Video Coding (HEVC) steganographic schemes, the cover object, such as motion vector and intra prediction mode, is coincident with H.264/MPEG-4 AVC. Consequently, the impact of new features introduced by HEVC on steganography security and capacity is still a question. In this paper, the steganographic properties of HEVC intra Coding Tree Unit (CTU) quadtree partition structure are studied.A steganography algorithm based on quadtree coding is proposed to embed the secret payload. Four kinds of quadtree coding embedding methods are designed in this algorithm to make full use of different sizes of Coding Blocks (CB).To minimize the potiential statistical detectability, we also propose an adaptive matching algorithm to make the statistical probability of stego CTU partition structure close to the actual one. Our results are better
than those obtained with current state of the art methods, and prove that steganography using intra CTU quadtree structure as cover object has higher capacity than those using intra prediction mode at the same bitrate variance ratio.

	
	
\end{abstract}	
\begin{IEEEkeywords}
		video steganography, HEVC, quadtree partition structure, adaptive steganography
\end{IEEEkeywords}
	
\section{Introduction\label{intro}}
	
Currently,



\section{Preliminaries}

\subsection{Quadtree Partition Structure}



\section{Steganography using the intra CTU quadtree partition structure}


\subsection{Overview of the Proposed algorithm}

\subsection{Analysis of statistical characteristic of different quadtree coding embedding methods}
\subsection{Quadtree coding embedding methods}

\subsection{Analysis of statistical characteristic of different quadtree coding embedding methods}

\subsection{Adaptive matching algorithm}


\section{Experiments}




	
	
	
	
	
	
	
\end{document}